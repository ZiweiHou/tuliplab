%
% ---------------------------------------------------------------
% Copyright (C) 2012-2018 Gang Li
% ---------------------------------------------------------------
%
% This work is the default powerdot-tuliplab style test file and may be
% distributed and/or modified under the conditions of the LaTeX Project Public
% License, either version 1.3 of this license or (at your option) any later
% version. The latest version of this license is in
% http://www.latex-project.org/lppl.txt and version 1.3 or later is part of all
% distributions of LaTeX version 2003/12/01 or later.
%
% This work has the LPPL maintenance status "maintained".
%
% This Current Maintainer of this work is Gang Li.
%
%

\documentclass[
 size=14pt,
 paper=smartboard,  %a4paper, smartboard, screen
 mode=present, 		%present, handout, print
 display=slides, 	% slidesnotes, notes, slides
 style=tuliplab,  	% TULIP Lab style
 pauseslide,
 fleqn,leqno]{powerdot}


\usepackage{amssymb}
\usepackage{amsmath}
\usepackage{rotating}
\usepackage{graphicx}
\usepackage{boxedminipage}
\usepackage{media9}
\usepackage{rotate}
\usepackage{calc}
\usepackage[absolute]{textpos}
\usepackage{psfrag,overpic}
\usepackage{fouriernc}
\usepackage{pstricks,pst-node,pst-text,pst-3d,pst-grad}
\usepackage{moreverb,epsfig,subfigure}
\usepackage{pstricks}
\usepackage{pstricks-add}
\usepackage{pst-text}
\usepackage{pst-node, pst-tree}
\usepackage{booktabs}
\usepackage{etex}
\usepackage{breqn}
\usepackage{multirow}
\usepackage{gitinfo2}
\usepackage{xcolor}

\usepackage{todonotes}
% \usepackage{pst-rel-points}

\usepackage{listings}
\lstset{frameround=fttt,
frame=trBL,
stringstyle=\ttfamily,
backgroundcolor=\color{yellow!20},
basicstyle=\footnotesize\ttfamily}
\lstnewenvironment{code}{
\lstset{frame=single,escapeinside=`',
backgroundcolor=\color{yellow!20},
basicstyle=\footnotesize\ttfamily}
}{}


\usepackage{hyperref}
\hypersetup{ % TODO: PDF meta Data
  pdftitle={Presentation Title},
  pdfauthor={Gang Li},
  pdfpagemode={FullScreen},
  pdfborder={0 0 0}
}


% \usepackage{auto-pst-pdf}
% package to show source code

\definecolor{LightGray}{rgb}{0.9,0.9,0.9}
\newlength{\pixel}\setlength\pixel{0.000714285714\slidewidth}
\setlength{\TPHorizModule}{\slidewidth}
\setlength{\TPVertModule}{\slideheight}
\newcommand\highlight[1]{\fbox{#1}}
\newcommand\icite[1]{{\footnotesize [#1]}}

\newcommand\twotonebox[2]{\fcolorbox{pdcolor2}{pdcolor2}
{#1\vphantom{#2}}\fcolorbox{pdcolor2}{white}{#2\vphantom{#1}}}
\newcommand\twotoneboxo[2]{\fcolorbox{pdcolor2}{pdcolor2}
{#1}\fcolorbox{pdcolor2}{white}{#2}}
\newcommand\vpspace[1]{\vphantom{\vspace{#1}}}
\newcommand\hpspace[1]{\hphantom{\hspace{#1}}}
\newcommand\COMMENT[1]{}

\newcommand\placepos[3]{\hbox to\z@{\kern#1
        \raisebox{-#2}[\z@][\z@]{#3}\hss}\ignorespaces}

\renewcommand{\baselinestretch}{1.2}


\newcommand{\draftnote}[3]{
	\todo[author=#2,color=#1!30,size=\footnotesize]{\textsf{#3}}	}
% TODO: add yourself here:
%
\newcommand{\gangli}[1]{\draftnote{blue}{GLi:}{#1}}
\newcommand{\shaoni}[1]{\draftnote{green}{sn:}{#1}}
\newcommand{\gliMarker}
	{\todo[author=GLi,size=\tiny,inline,color=blue!40]
	{Gang Li has worked up to here.}}
\newcommand{\snMarker}
	{\todo[author=Sn,size=\tiny,inline,color=green!40]
	{Shaoni has worked up to here.}}

%%%%%%%%%%%%%%%%%%%%%%%%%%%%%%%%%%%%%%%%%%%%%%%%%%%%%%%%%%%%%%%%%%%%%%%%
% title
% TODO: Customize to your Own Title, Name, Address
%
\title{Team for Universal Learning and Intelligent Processing}
\author{
Gang Li
\\
\\School of Information Technology
\\Deakin University, Australia
}
\date{\gitCommitterDate}


% Customize the setting of slides
\pdsetup{
% TODO: Customize the left footer, and right footer
rf=\href{http://www.tulip.org.au}{
Last Changed by: \textsc{\gitCommitterName}\ \gitVtagn-\gitAbbrevHash\ (\gitAuthorDate)
},
cf={Recent Research \@ TULIP Lab},
}


\begin{document}

\maketitle

%\begin{slide}{Overview}
%\tableofcontents[content=sections]
%\end{slide}


%%==========================================================================================
%%
\begin{slide}[toc=,bm=]{Overview}
\tableofcontents[content=currentsection,type=1]
\end{slide}
%%
%%==========================================================================================


\section{TULIP Lab}


%%==========================================================================================
%%
\begin{slide}{TULIP Team}
\begin{center}
\twotonebox{\rotatebox{90}{Defn}}{\parbox{.86\textwidth}
{Outlying Aspects Mining aims to
identify the outstanding features of the query object.
\begin{itemize}
\item A teacher may be interested in the \textcolor{orange}{characteristics} that
make \textcolor{orange}{one student} \textcolor{orange}{distinctive} from others.
\item NBA coaches would prefer to
find out the strengths and weaknesses of the player (a query object).
\end{itemize}
}}

\end{center}
\bigskip
\begin{center}
\begin{tabular}{c| c c c c }
\toprule
Player & \texttt{3PT\%}  & \texttt{FTA} & \texttt{FT\%} & \texttt{To} \\
\midrule
$P_1$
&  {$65$} &  {$4$} &  {$33$} &  {$8$} \\
$P_2$
&  {$78$} &  {$1$}&  {$65$}&  {$5$} \\
$P_3$
&  {$58$} &  {$6$} &  {$46$} &  {$3$} \\
$P_4$
&  {$68$} &  {$1.2$}&  {$85$}&  {$6.2$} \\
$P_5$
&  {$58$} &  {$6.2$} &  {$36$} &  {$3.4$}\\
\bottomrule
\end{tabular}
\end{center}

\end{slide}
%%
%%==========================================================================================


%%==========================================================================================
%%
\begin{slide}{TULIP Lab}
\begin{center}
\begin{tabular}{c| c c c c }
\toprule
%\centering
Player & \texttt{3PT\%}  & \texttt{FTA} & \texttt{FT\%} & \texttt{To} \\
\midrule
$P_1$
&  {$65$} &  {$4$} &  {$33$} &  {$8$} \\
$P_2$
&  {$78$} &  {$1$}&  {$65$}&  {$5$} \\
$P_3$
&  {$58$} &  {$6$} &  {$46$} &  {$3$} \\
$P_4$
&  {$68$} &  {$1.2$}&  {$85$}&  {$6.2$} \\
$P_5$
&  {$58$} &  {$6.2$} &  {$36$} &  {$3.4$}\\
\bottomrule
\end{tabular}
\end{center}

\bigskip

\twocolumn[
\savevalue{lfrheight}=4.6cm,
\savevalue{lfrprop}={
linestyle=solid,framearc=.2,linewidth=1pt},
rfrheight=\usevalue{lfrheight},
rfrprop=\usevalue{lfrprop}
]{
Outlying Aspects Mining
\begin{itemize}
\item
\smallskip
Explain the distinctive \textcolor{orange}{aspects} of the query object.
\smallskip
\item
\smallskip
The query object may (or may not) be an outlier.
\end{itemize}
}{
Outlier Detection
\begin{itemize}
\item
\smallskip
Find out \textcolor{orange}{all} unusual
\textcolor{orange}{objects} in the whole dataset.
\smallskip
\item
\smallskip
\textcolor{orange}{No} explanation on how they are different.
\end{itemize}
}

\end{slide}
%%
%%==========================================================================================


%%==========================================================================================
%%
\begin{slide}{Lab Organization}
\twotonebox {\rotatebox{90}{Defn}}{\parbox{.88\textwidth}
{
{\textcolor{orange}{Group outlying aspects mining} aims to
identify the outstanding features of the group of query object.
\begin{itemize}
\item
Doctors desire to identify the merits \& demerits between
\textcolor{orange}{a group of cancer patients} and normal people.
\item
NBA coaches are passionate about exploring the obvious advantages \&
disadvantages of \textcolor{orange}{the team}.
\end{itemize}
}
}}

\vspace{1.5cm}

\twocolumn{
\begin{figure}
  \centering
  \selectcolormodel{rgb}
  \missingfigure{Testing.}
  %\includegraphics[width=0.6\textwidth]{figures//demical.eps}\\
  \caption{Medical}\label{fig:demical}
\end{figure}
}{
\begin{figure}
  \centering
  \selectcolormodel{rgb}
  \missingfigure{Testing.}
  %\includegraphics[width=0.6\textwidth]{figures//NBA_team.eps}\\
  \caption{NBA-Team}\label{fig:timg}
\end{figure}
}

\end{slide}
%%
%%==========================================================================================


\section{Research at TULIP}


%%==========================================================================================
%%
\begin{slide}[toc=,bm=]{Research at TULIP}
\begin{itemize}
\item
Existing Methods - \textcolor{orange}{Feature selection}

\begin{itemize}
\item
To distinguish two classes:
the query point (positive) \& rest of data (negative)
\end{itemize}
\vspace{1cm}
\twocolumn[
\savevalue{lfrheight}=5cm,
\savevalue{lfrprop}={
linestyle=solid,framearc=.2,linewidth=1pt},
rfrheight=\usevalue{lfrheight},
rfrprop=\usevalue{lfrprop}
]{
Disadvantages
\begin{itemize}
\item
\smallskip
Positive and negative classes are \textcolor{orange}{Not} balanced.

\item
\smallskip
\textcolor{orange}{Not} quantify the outlying degree accurately.

\item
\smallskip
\textcolor{orange}{Not} identify group outlying aspects.
\end{itemize}
}
{
Advantages
\begin{itemize}
\item
\smallskip
Easy to operate.

\item
\smallskip
Resolve dimensionality bias.
\end{itemize}
}
\end{itemize}

\end{slide}
%%
%%==========================================================================================


%%==========================================================================================
%%
\begin{slide}{Theme one - Behavior Informatics}

\begin{itemize}
\item
Existing Methods - \textcolor{orange} {Score-and-search}

\begin{itemize}
\item
Define an outlying score function.

\item
Search subspaces.
\end{itemize}
\bigskip
\twocolumn[
\savevalue{lfrheight}=5cm,
\savevalue{lfrprop}={
linestyle=solid,framearc=.2,linewidth=1pt},
rfrheight=\usevalue{lfrheight},
rfrprop=\usevalue{lfrprop}
]{
Disadvantages
\begin{itemize}
\item
\smallskip
Dimensionality bias.

\item
\smallskip
Search efficiency is \textcolor{orange}{Not} high (dataset is large).

\item
\smallskip
\textcolor{orange}{Not} identify group outlying aspects.
\end{itemize}
}{
Advantages
\begin{itemize}
\item
\smallskip
Quantify the outlying degree correctly.

\item
\smallskip
High Comprehensibility.

\end{itemize}
}
\end{itemize}

\end{slide}
%%
%%==========================================================================================


%%==========================================================================================
%%
\begin{slide}[toc=,bm=]{The Popularity of Online Social Networks}


\end{slide}
%%
%%==========================================================================================


%%==========================================================================================
%%
\begin{slide}[toc=,bm=]{Application of Geo Photos}


\end{slide}
%%
%%==========================================================================================


%%==========================================================================================
%%
\begin{slide}[toc=,bm=]{Tourist Movement Analysis}


\end{slide}
%%
%%==========================================================================================


%%==========================================================================================
%%
\begin{slide}[toc=,bm=]{Tourist Movement Analysis}


\end{slide}
%%
%%==========================================================================================


%%==========================================================================================
%%
\begin{slide}[toc=,bm=]{Tourist Movement Analysis}
\twocolumn
{
Group Outlying Aspects Mining
\begin{itemize}
\item
\smallskip
Focus on differences between \textcolor{orange}{groups}.

\item
\smallskip
\textcolor{orange}{Multiple} points.
\medskip
\end{itemize}
\vspace{0.75cm}
%\vspace{0.1cm}
\begin{figure}
  \centering
  \selectcolormodel{rgb}
  \missingfigure{Testing a long text string.}
  %\includegraphics[width=0.6\textwidth]{figures//example-basketball-projection.eps}\\
  \caption{Group Outlying Aspects Target}\label{fig:GroupOutAspect-target}
\end{figure}
}
{
Outlying Aspects Mining
\begin{itemize}
\item
Concentrates on differences between \textcolor{orange}{objects}.

\item
\textcolor{orange}{One} point.
\end{itemize}
\bigskip
\begin{figure}
  \centering
  \selectcolormodel{rgb}
  \missingfigure{Testing a long text string.}
%  \includegraphics[width=0.5\textwidth]{figures//OutAspect_target.eps}\\
  \caption{Outlying Aspects Target}\label{fig:OutAspect-target}
\end{figure}
}

\end{slide}
%%
%%==========================================================================================


%%==========================================================================================
%%
\begin{slide}[toc=,bm=]{Periodic Behavior Mining}


\end{slide}
%%
%%==========================================================================================


%%==========================================================================================
%%
\begin{slide}[toc=,bm=]{Periodic Behavior Mining}


\end{slide}
%%
%%==========================================================================================


%%==========================================================================================
%%
\begin{slide}[toc=,bm=]{Periodic Behavior Mining Mobility Intentions}


\end{slide}
%%
%%==========================================================================================


%%==========================================================================================
%%
\begin{slide}[toc=,bm=]{Periodic Behavior Mining Applications}


\end{slide}
%%
%%==========================================================================================


%%==========================================================================================
%%
\begin{slide}[toc=,bm=]{P.R. Applications - Case 1: Face Age Recognition}


\end{slide}
%%
%%==========================================================================================


%%==========================================================================================
%%
\begin{slide}[toc=,bm=]{P.R. Applications - Case 2: Speech-based Emotion Detection}


\end{slide}
%%
%%==========================================================================================


%%==========================================================================================
%%
\begin{slide}[toc=,bm=]{P.R. Applications - Case 3: K-Complex Detection}


\end{slide}
%%
%%==========================================================================================


%%==========================================================================================
%%
\begin{slide}[toc=,bm=]{P.R. Applications - Case 4: Trajectory Analysis}


\end{slide}
%%
%%==========================================================================================


%%==========================================================================================
%%
\begin{slide}[toc=,bm=]{P.R. Applications - Case 5: Video Surveillance}


\end{slide}
%%
%%==========================================================================================


%%==========================================================================================
%%
\begin{slide}[toc=,bm=]{Market Segmentation Analysis}


\end{slide}
%%
%%==========================================================================================


%%==========================================================================================
%%
\begin{slide}[toc=,bm=]{Negative Association Rule Mining}


\end{slide}
%%
%%==========================================================================================


%%==========================================================================================
%%
\begin{slide}[toc=,bm=]{Contrast Mining}


\end{slide}
%%
%%==========================================================================================


%%==========================================================================================
%%
\begin{slide}{Theme two - Information Abuse Prevention}
\begin{itemize}
\item
How to \textcolor{orange}{represent} the group features.

\begin{itemize}
\item
Can be affected by outlier values.

\item
Can \textcolor{orange}{Not} reflect the overall distribution of group features.
\end{itemize}
\end{itemize}

\end{slide}
%%
%%==========================================================================================


%%==========================================================================================
%%
\begin{slide}[toc=,bm=]{Who is an adversary?}

\begin{itemize}
\item
How to \textcolor{orange}{evaluate} the outlying degree in different aspects.

\begin{itemize}
\item
Need design a scoring function when necessary.

\item
Adopting an appropriate scoring function (without dimension bias) remains a problem.

\end{itemize}
\end{itemize}

\end{slide}
%%
%%==========================================================================================


%%==========================================================================================
%%
\begin{slide}[toc=,bm=]{Privacy Breach Example (1)}

\begin{itemize}
\item
How to \textcolor{orange}{improve} the efficiency.

\begin{itemize}

\item
When the dimension of the \textcolor{orange}{data is high},
the candidate subspace grows exponentially.

\item
It will easily go beyond the limits of the computation resources.

\end{itemize}
\end{itemize}

\end{slide}
%%
%%==========================================================================================


%%==========================================================================================
%%
\begin{slide}[toc=,bm=]{Privacy Breach Example (2)}


\end{slide}
%%
%%==========================================================================================


%%==========================================================================================
%%
\begin{slide}[toc=,bm=]{Is O.S.N. a Secure Place to Show off?}


\end{slide}
%%
%%==========================================================================================


%%==========================================================================================
%%
\begin{slide}[toc=,bm=]{I Know Where Your Cat Lives}


\end{slide}
%%
%%==========================================================================================


%%==========================================================================================
%%
\begin{slide}[toc=,bm=]{AOL Dataset Debacle}


\end{slide}
%%
%%==========================================================================================


%%==========================================================================================
%%
\begin{slide}[toc=,bm=]{Breach of medical record}


\end{slide}
%%
%%==========================================================================================


%%==========================================================================================
%%
\begin{slide}[toc=,bm=]{Breach of medical record}


\end{slide}
%%
%%==========================================================================================


%%==========================================================================================
%%
\begin{slide}[toc=,bm=]{Privacy Preserving Related Reference}


\end{slide}
%%
%%==========================================================================================


%%==========================================================================================
%%
\begin{slide}[toc=,bm=]{Our rankings}


\end{slide}
%%
%%==========================================================================================


\section{Training at TULIP}


%%==========================================================================================
%%
\begin{slide}{Training at TULIP}

Framework of GOAM algorithm:

\bigskip

\begin{figure}
  \centering
  \selectcolormodel{rgb}
  \missingfigure{Testing a long text string.}
%  \includegraphics[width=0.55\textwidth]{figures//framework1.eps}\\
  \caption{Framework of GOAM Algorithm} \label{framework}
\end{figure}

\end{slide}
%%
%%==========================================================================================


%%==========================================================================================
%%
\begin{slide}[toc=,bm=]{Training at TULIP}



\end{slide}
%%
%%==========================================================================================


%%==========================================================================================
%%
\begin{slide}{Moving to Higher Stages}
\begin{itemize}
\item
\smallskip
Suppose $f_1$, $f_2$, $f_3$ are three features of $G_q$.

$f_1$: \{$x_1, x_2, x_3, x_4, x_5, x_2, x_3, x_4, x_1, x_2$\} \\

$f_2$: \{$y_2, y_2, y_1, y_2, y_3, y_3, y_5, y_4, y_4, y_2$\} \\

$f_3$: \{$z_1, z_4, z_2, z_4, z_5, z_3, z_1, z_2, z_4, z_2$\} \\
\end{itemize}

\begin{figure}[htbp]
    \centering
    \subfigure[$f_1$]{
        \selectcolormodel{rgb}
        \missingfigure[figwidth=5.5cm]{Test.}
        %\includegraphics[width=0.25\textwidth]{figures//frequency-distribution-feature1.eps}
        \label{fig:fre-dis-f1}
    }
    \subfigure[$f_2$]{
        \selectcolormodel{rgb}
        \missingfigure[figwidth=5.5cm]{Test.}
        \label{fig:fre-dis-f2}
    }
    \subfigure[$f_3$]{
        \selectcolormodel{rgb}
        \missingfigure[figwidth=5.5cm]{Test.}
        \label{fig:fre-dis-f3}
    }
    \caption{Histogram of $G_q$ on three features}
    \label{fig:fre-dis-each-feature}
\end{figure}

\end{slide}
%%
%%==========================================================================================


\section{Rules in TULIP}


%%==========================================================================================
%%
\begin{slide}{Regulations}

\begin{center}
\begin{itemize}

\item<1->
{TULIP Academy}

\begin{itemize}
\item
You must not be absent from the meeting without permission.

\item
Xi'an group members must attend the seminar in the Lab,
and NOT online.

\item
Collaborate with team members, rather than attacking each other.
\end{itemize}

\item<2->
{FLIP Study}

\begin{itemize}
\item
All FLIP students need to pass the corresponding prerequisite.

\item
All FLIP materials are confidential, and you should not circulate it outside.
\end{itemize}

\item<3->
{Research}

\begin{itemize}
\item
SmartGit, Bitbucket, Mendeley, update PPR in time.

\item
Actively discuss research work in the group meeting, and TULIP Seminar.

\item
Everyone should prepare at least one question for each TULIP seminar.
\end{itemize}
\end{itemize}
\end{center}

\end{slide}
%%
%%==========================================================================================


%%==========================================================================================
%
\begin{slide}[toc=,bm=]{Questions?}



\end{slide}
%%
%%==========================================================================================


%%==========================================================================================
% TODO: Contact Page
\begin{wideslide}[toc=,bm=]{Contact Information}
\centering
\vspace{\stretch{1}}
\twocolumn[
lcolwidth=0.35\linewidth,
rcolwidth=0.65\linewidth
]
{
% \centerline{\includegraphics[scale=.2]{tulip-logo.eps}}
}
{
\vspace{\stretch{1}}
Associate Professor \emph{Gang Li}\\
School of Information Technology\\
Deakin University, Geelong, Australia
\begin{description}
 \item[Email] \href{mailto:gangli@acm.org}
 {\textsc{\footnotesize{gangli@acm.org}}}

 \item[Lab] \href{http://www.tulip.org.au}
 {\textsc{\footnotesize{Team for Universal Learning and Intelligent Processing}}}
\end{description}
}
\vspace{\stretch{1}}
\end{wideslide}

\end{document}

\endinput
